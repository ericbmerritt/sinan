\documentclass[11pt]{article}
\usepackage{minted}
\usemintedstyle{monokai}
\begin{document}
\title{Sinan Manual}
\author{Eric Merritt}
\maketitle
\tableofcontents
\addcontentsline{toc}{section}{References}

\section {Introduction}
\section {Generating a Project}
\section {Integrating With External Tools}

Sinan provides a mechenism to integrate with external tools. This is
via the hooks functionality. This functionality is used by putting a
directory called \verb|_hooks| in the root of your project. Inside
this directory you place executable files that implement the
functionality you want to run. When these files are run is determenid
by there name. The are named in the form of
\verb|<when>-<build_task>|. The when is may be one of two values,
either \verb|pre| or \verb|post|. The build task is the name of the
task you want the think to run around. For example, if you wanted
something to run just after the build task is complete you would
create a file called \verb|post-build| and make it executable. This
file will then be run (in the context of the root directory)
immediatly after the build task and before any other tasks or run. A
file called \verb|pre-build| would do just the opposite, running
imeditaly before the build task.

Having the ability to run these files at specific times is
good. However, without getting some information from sinan its not
terribly useful. Fortunatly, there is a mechenism for getting this
information. Sinan passes it to the running script via environmental
variables. There are two sets of environmental variables. The first is
the global information about the project. These variables are
available as follows.

\begin{description}
\item[PREFIX] The file path of the erlang/erlware installation that
  sinan is running out of.
\item[ERTS\_VSN] The erts version that sinan is building of
\item[BUILD\_DIR] The build directory where the projectis being built
  too
\item[BUILD\_FLAVOR] The current build flavor of the system
\item[BUILD\_REF] The unique id for this run of sinan
\item[PROJECT\_DIR] The project root directory
\item[PROJECT\_NAME] The project name as defined in the build config
\item[PROJECT\_VSN] The project version
\item[PROJECT\_APPS] A comma seperated list of apps that are part of
  the project.
\item[PROJECT\_DEPS] A comma seperated list of apps that are
  dependencies of the project
\end{description}

The second set environmental variables that are available for each
application and dependency in the project. The 'APP' in the following
descriptions is replaced by the name of the app in upper case. These
values are as follows.

\begin{description}
\item[APP\_VSN] The version of the application
\item[APP\_LOCATION] The file path to the application that sinan
  is usig.
\item[APP\_DEPS] A comma seperated list of the names of the
  dependent applicatinos for this application.
\end{description}

The combination of scripts and environmental variables should be
enough to do most required actions on the these tertiary builds.



\end{document}
